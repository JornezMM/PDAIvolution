\capitulo{2}{Objetivos del proyecto}

El principal objetivo de este proyecto es la creación de una aplicación web que permita, mediante la evaluación del grado de bradicenesia utilizando vídeos de la prueba del <<rapid finger tapping test>> y el uso de inteligencia artificial realizar dicha evaluación. Además, esta aplicación debe poder ser usada por el usuario promedio.


\section{Objetivos técnicos}

Por ello, para poder alcanzar estos objetivos generales, se plantean los siguientes objetivos técnicos:
\begin{itemize}
\item \textbf{Revisar de trabajos relacionados}: para comprender mejor el campo que se esta investigando y obtener conocimientos relacionados con el tema.
\item \textbf{Desarrollar el modelo de aprendizaje supervisado}: entrenar, evaluar y seleccionar el modelo que mejor resultados ofrezca.
\item \textbf{Evaluar la calidad de la solución}: comprobando si los resultados que nos ofrece son coherentes.
\item \textbf{Crear de una aplicación \textit{web}}: que sea fácil de usar e intuitiva para el usuario.
\item \textbf{Familiarizarse con nuevas tecnologías}: como programación \textit{web} o Docker.
\item \textbf{Documentar el proyecto}: de manera clara y concisa, condensando toda la información relevante en un formato accesible y fácil de entender.
\end{itemize}

\section{Objetivos de \textit{software}}

Se han propuesto los siguiente objetivos de desarrollo de \textit{software}.
\begin{itemize}
\item \textbf{Crear una interfaz de usuario accesible}: para que el usuario medio pueda usarla sin dificultad.
\item \textbf{Crear un sistema de gestión de datos persistente}: como bases de datos SQL.
\item \textbf{Asegurar la calidad del código}: mediante herramientas de análisis de código como SonarCloud.
\item \textbf{Deplegar en plataformas}: mediante el uso de herramientas como Docker, permitiendo además la familiarización con estas tecnologías.
\item \textbf{Implementar trabajos anteriores}: para realizar la extracción de características y obtener pautas para el correcto desarrollo.
\end{itemize}