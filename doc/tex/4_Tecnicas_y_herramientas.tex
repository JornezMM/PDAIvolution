\capitulo{4}{Técnicas y herramientas}

En este capítulo se muestran las diferentes técnicas y herramientas que se han utilizado para el desarrollo del proyecto.


\section{Técnicas}

En esta sección se muestran las técnicas principales empleadas para el desarrollo del proyecto.

\subsection{Scrum}

Para el desarrollo de este proyecto ha sido utilizada la metodología ágil conocida como Scrum.

La metodología Scrum trata de fraccionar la duración de un proyecto en lo que se conoce como <<sprints>>, cuya duración varía de una a dos semanas.
El objetivo de los <<sprints>> es decidir que parte del proyecto se va a desarrollar durante ese periodo de tiempo realizando revisiones diarias para ver como avanza asi como una revisión de <<sprint>> en el que hace una valoración general del <<sprint>> y se decide en que va a consistir el siguiente.

\section{Herramientas}

En esta sección se describen las herramientas que se han utilizado durante la realización del proyecto.

\subsection{Zube}

Es una herramienta que nos permite la gestión del proyecto usando la metodología Scrum. Esta herramienta nos ayuda a la creación de <<sprints>>, así como la visualización del progreso del <<sprint>>.


\subsection{Git}
Git es una herramienta de control de versiones que permite llevar un registro de los cambios realizados en archivos así como mantener un historial completo de las actualizaciones.


\subsection{\TeX{}Maker}
Editor de \LaTeX{} utilizado.

\subsection{DrawIo}
Es una herramienta que permite la creación de diagramas entidad-relación, diagramas de flujo, casos de uso, etc. Esta herramienta además dispone de los símbolos UML necesarios para todas las funcionalidades.