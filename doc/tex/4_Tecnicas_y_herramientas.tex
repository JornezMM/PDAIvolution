\capitulo{4}{Técnicas y herramientas}

En este capítulo se muestran las diferentes técnicas y herramientas que se han utilizado para el desarrollo del proyecto.


\section{Técnicas}

En esta sección se muestran las técnicas principales empleadas para el desarrollo del proyecto.

\subsection{Scrum}

Para el desarrollo de este proyecto ha sido utilizada la metodología ágil conocida como Scrum.

La metodología Scrum trata de fraccionar la duración de un proyecto en lo que se conoce como <<sprints>>, cuya duración varía de una a dos semanas.
El objetivo de los <<sprints>> es decidir que parte del proyecto se va a desarrollar durante ese periodo de tiempo realizando revisiones diarias para ver como avanza asi como una revisión de <<sprint>> en el que hace una valoración general del <<sprint>> y se decide en que va a consistir el siguiente.

\section{Herramientas}

En esta sección se describen las herramientas que se han utilizado durante la realización del proyecto.

\subsection{Portales}

Usados principalmente para el seguimiento de la metodología Scrum, creación de diagramas y bocetos y control de versiones:
\begin{itemize}
\item \textbf{Zube}: portal dedicado a la gestión de proyectos <<software>>~\cite{zubeHome}. Permite visualizar proyectos, sprints,
gráficos propios de Scrum y crear tableros.
\item \textbf{GitHub}: es una plataforma web que proporciona una interfaz gráfica para el control de versiones utilizando Git~\cite{gitbubHome}. Además, GitHub permite a los usuarios revisar versiones anteriores de proyectos y realizar un seguimiento de los cambios a lo largo del tiempo.
\item \textbf{DrawIO}: es una herramienta que permite la creación de diagramas entidad-relación, diagramas de flujo, casos de uso, etc~\cite{drawioHome}. Esta herramienta dispone además de los símbolos UML necesarios para todas las funcionalidades.
\end{itemize}

\subsection{Librerías}
Principalmente de Python:
\begin{itemize}
\item \textbf{Scikit-Learn}: esta librería ofrece una amplia variedad de algoritmos de aprendizaje supervisados, semisupervisados y no supervisados~\cite{sklearnHome}. Además posee una gran compatibilidad con otras librerías como así como una extensa documentación que facilitan su implementación.3.
\item \textbf{TSFresh}: es una biblioteca de Python, para la extracción y representación de características de series temporales~\cite{tsfreshHome}. Esta herramienta agiliza el proceso de extracción de características a partir de datos de series temporales para apoyar las tareas de aprendizaje automático. Ofrece una selección de características definidas que abarcan diferentes medidas estadísticas, como análisis de tendencias, patrones estacionales, métricas de correlación y evaluaciones de complejidad.
\end{itemize}
