\apendice{Plan de Proyecto Software}

\section{Introducción}
El propósito de este documento es establecer una guía de forma clara y organizada para la ejecución de un proyecto. En él se definen los objetivos y la forma en la que se van a alcanzar.


\subsection{Scrum}
Para la planificación del proyecto ha sido empleado el marco de gestión de proyectos de metodología ágil~\cite{scrumMaster2022}. Scrum es una metodología ágil para la gestión de proyectos complejos en entornos cambiantes. Se basa en tres pilares: eventos, roles y artefactos, y se trabaja en sprints de una duración determinada, generalmente entre una semana y un mes. Los equipos de Scrum se autoorganizan y se comprometen a entregar resultados de alta calidad de manera eficiente y creativa.
\subsubsection{Roles}
En este marco de gestión de proyectos, se destacan tres roles principales:

\begin{itemize}
\item \textsl{Scrum Master}: Líder del equipo encargado de eliminar obstáculos y facilitar la auto-organización y coordinación del equipo.
\item \textsl{Product Owner}: Representa la voz del cliente, lidera el desarrollo del producto y busca el valor para los usuarios.
\item Equipo de desarrollo: Responsable de ejecutar las acciones previstas para el éxito del proceso
\end{itemize}
\subsubsection{Eventos}
Los eventos clave en Scrum son:
\begin{itemize}


\item \textsl{Sprint}: Reunión para planificar el trabajo del próximo sprint.
\item Scrum diario: Breve reunión diaria para sincronizar actividades.
\item Revisión del \textsl{sprint}: Revisión del producto al final del \textsl{sprint}.
\item Retrospectiva del \textsl{sprint} : Análisis de los éxitos y fallos del \textsl{sprint}
\end{itemize}




\subsubsection{Artefactos}
Scrum utiliza tres artefactos principales:
\begin{itemize}
\item Pila de producto o \textsl{product backlog}: Inventario que contiene el trabajo pendiente en el producto.
\item Pila del \textsl{sprint} o \textsl{sprint backlog}: Lista de tareas a realizar durante el \textsl{sprint}.
\item Incremento: Versión mejorada y funcional del producto al final de cada \textsl{sprint}.
\end{itemize}




 
\section{Planificación temporal}
\subsection{Planificación mediante \textsl{sprints}}
Se ha decido que la planificación del proyecto se haga mediante \textsl{sprints} debido al tamaño del equipo (un desarrollador).


\subsubsection{Sprint 1}
\begin{itemize}
\item \textbf{Objetivos}
\begin{enumerate}
\item Configuración inicial: creación del repositorio y configuración de cuenta de ZenHub.
\item Memoria: redacción de la introducción y familarización con las aplicaciones para edición de archivos \TeX .
\item Lectura de papers:  <<Supervised classification of bradykinesia in Parkinson’s disease from smartphone videos>>  \cite{williams2020discerning}, <<The discerning eye of computer vision: Can it measure Parkinson's finger tap bradykinesia?>> \cite{williams2020supervised}  y <<A computer vision framework for finger-tapping evaluation in Parkinson's disease>> \cite{khan2014computer}.
\end{enumerate}
\item \textbf{Periodo}
Este \textsl{sprint} se desarrolló entre el 2 de octubre del 2023 y el 16 de octubre del 2023.
\item \textbf{\textsl{Review}}
En la revisión se resolvieron dudas sobre la documentación y se decidió cambiar de herramienta para la realización de la metodología scrum del proyecto a zube. 


\end{itemize}

\subsubsection{Sprint 2}
\begin{itemize}
\item \textbf{Objetivos}
\begin{enumerate}
\item Documentación de trabajos previos: lectura del trabajo realizado por Catalin para ver la extracción de datos.
\item Memoria: redacción del apartado de trabajos relacionados.
\item Curso de flask: realización de los tutoriales del framework de flask para la realización del apartado web.
\end{enumerate}
\item \textbf{Periodo}
Este \textsl{sprint} se desarrolló entre el 16 de octubre del 2023 y el 29 de octubre del 2023.
\item \textbf{\textsl{Review}}
Se resolvieron dudas sobre dónde encontrar la información dentro del proyecto de Catalin así como problemas a la hora de compilar los documentos de \LaTeX . 


\end{itemize}
\subsubsection{Sprint 3}
\begin{itemize}
\item \textbf{Objetivos}
\begin{enumerate}
\item Documentación de trabajos previos: lectura de los documentos para la extracción de datos usando el paquete de python TSFresh.
\item Memoria: corrección de errores en la documentación y añadida biografía.
\item Curso de flask: continuada la realización de los tutoriales del framework de flask para la realización del apartado web.
\item Repositorio: añadido el archivo gitignore para evitar la subida de archivos temporales.
\end{enumerate}
\item \textbf{Periodo}
Este \textsl{sprint} se desarrolló entre el 30 de octubre del 2023 y el 12 de noviembre del 2023.
\item \textbf{\textsl{Review}}
Problemas a la hora de la realización de los objetivos propuestos agregados en el siguiente \textsl{sprint}. 


\end{itemize}

\subsubsection{Sprint 4}
\begin{itemize}
\item \textbf{Objetivos}
\begin{enumerate}
\item Creación del \textit{mockup} de la aplicación.
\item Documentación de los anexos.
\end{enumerate}
\item \textbf{Periodo}
Este \textsl{sprint} se desarrolló entre el 13 de noviembre del 2023 y el 30 de diciembre del 2023.
\item \textbf{\textsl{Review}}
Se revisó el \textit{mockup} y se propuso una modificación del mismo. 


\end{itemize}


\subsubsection{Sprint 5}
\begin{itemize}
\item \textbf{Objetivos}
\begin{enumerate}
\item Finalización del curso de flask.
\item Modificación del \textit{mockup}.
\item Creación de diagramas entidad relación y diagramas de casos de uso.
\end{enumerate}
\item \textbf{Periodo}
Este \textsl{sprint} se desarrolló entre el 30 de noviembre del 2023 y el 12 de diciembre del 2023.
\item \textbf{\textsl{Review}}
Se revisaron tanto los diagramas como el \textit{mockup} y se propusieron unos cambios en los diagramas. 


\end{itemize}

\subsubsection{Sprint 6}
\begin{itemize}
\item \textbf{Objetivos}
\begin{enumerate}
\item Documentación general añadiendo técnicas y herramientas.
\item Búsqueda de trabajos parecidos o aplicaciones médicas similares.
\item Modificación de los diagramas de casos de uso y entidad-relación.
\end{enumerate}
\item \textbf{Periodo}
Este \textsl{sprint} se desarrolló entre el 13 de diciembre del 2023 y el 19 de enero de 2024.
\item \textbf{\textsl{Review}}
Se cumplieron todos los objetivos del sprint. 


\end{itemize}
\subsubsection{Sprint 7}
\begin{itemize}
\item \textbf{Objetivos}
\begin{enumerate}
\item Documentación general añadiendo técnicas y herramientas.
\item Búsqueda de trabajos parecidos o aplicaciones médicas similares.
\item Modificación del diagramas de casos de uso y entidad-relación.
\end{enumerate}
\item \textbf{Periodo}
Este \textsl{sprint} se desarrolló entre el 20 de enero de 2024 y el 15 de febrero de 2024.
\item \textbf{\textsl{Review}}
Se cumplieron todos los objetivos del sprint. Se detectaron ciertos fallos en la documentación que se añadieron como tarea para el siguiente sprint. 


\end{itemize}
\subsubsection{Sprint 8}
\begin{itemize}
\item \textbf{Objetivos}
\begin{enumerate}
\item Comienzo de la página web.
\item Conceptos teóricos de la aplicación.
\item Solución de errores detectados en la anterior revisión.
\item Conexión con la base de datos.
\end{enumerate}
\item \textbf{Periodo}
Este \textsl{sprint} se desarrolló entre el 16 de febrero del 2024 y el 29 de febrero de 2024.
\item \textbf{\textsl{Review}}
Se cumplieron todos los objetivos del sprint. 
\end{itemize}

\subsubsection{Sprint 9}
\begin{itemize}
\item \textbf{Objetivos}
\begin{enumerate}
\item Continuación de la redacción de los conceptos teóricos.
\item Creación de la página de registro.
\end{enumerate}
\item \textbf{Periodo}
Este \textsl{sprint} se desarrolló entre el 1 de marzo del 2024 y el 14 de marzo de 2024.
\item \textbf{\textsl{Review}}
No se cumplió el objetivo de la documentación que se añadió al siguiente sprint. 
\end{itemize}

\subsubsection{Sprint 10}
\begin{itemize}
\item \textbf{Objetivos}
\begin{enumerate}
\item Continuación de la redacción de los conceptos teóricos.
\item Creación de la página de inicio de sesión.
\end{enumerate}
\item \textbf{Periodo}
Este \textsl{sprint} se desarrolló entre el 15 de marzo del 2024 y el 8 de abril de 2024.
\item \textbf{\textsl{Review}}
Se cumplieron todos los objetivos del sprint.  
\end{itemize}

\subsubsection{Sprint 11}
\begin{itemize}
\item \textbf{Objetivos}
\begin{enumerate}
\item Documentación de técnicas y herramientas.
\item Creación de la página principal de administrador.
\item Solución de errores relacionados con la página de registro.
\item Creación de la página de modificación de usuarios.
\end{enumerate}
\item \textbf{Periodo}
Este \textsl{sprint} se desarrolló entre el 9 de abril de 2024 y el 25 de abril de 2024.
\item \textbf{\textsl{Review}}
Se cumplieron todos los objetivos del sprint. 
\end{itemize}

\subsubsection{Sprint 12}
\begin{itemize}
\item \textbf{Objetivos}
\begin{enumerate}
\item Creación de la página principal del paciente.
\item Añadir la funcionalidad de actualizar el usuario en la base de datos una vez modificado.
\item Iniciar entrenamiento de los modelos para la predicción de características.
\end{enumerate}
\item \textbf{Periodo}
Este \textsl{sprint} se desarrolló entre el 26 de abril de 2024 y el 8 de mayo de 2024.
\item \textbf{\textsl{Review}}
No se pudieron realizar los experimentos debido a problemas de compatibilidad de versión. 
\end{itemize}

\subsubsection{Sprint 13}
\begin{itemize}
\item \textbf{Objetivos}
\begin{enumerate}
\item Creación de la página principal del doctor.
\item Continuar con el entrenamiento de los modelos.
\end{enumerate}
\item \textbf{Periodo}
Este \textsl{sprint} se desarrolló entre el 9 de mayo de 2024 y el 15 de mayo de 2024.
\item \textbf{\textsl{Review}}
Debido a problemas durante el entrenamiento de los modelos, no se pudo completar este objetivo correctamente. 
\end{itemize}

\subsubsection{Sprint 14}
\begin{itemize}
\item \textbf{Objetivos}
\begin{enumerate}
\item Continuar con el entrenamiento de los modelos.
\end{enumerate}
\item \textbf{Periodo}
Este \textsl{sprint} se desarrolló entre el 16 de mayo de 2024 y el 22 de mayo de 2024.
\item \textbf{\textsl{Review}}
Se cumplieron todos los objetivos del sprint y se consiguió resolver el problema con el entrenamiento de modelos.
\end{itemize}

\subsubsection{Sprint 15}
\begin{itemize}
\item \textbf{Objetivos}
\begin{enumerate}
\item Agregar funcionalidad de gestionar medicinas de los pacientes.
\item Agregar funcionalidad de ver medicinas del paciente.
\item Modificar la base de datos para añadir medicinas a la aplicación.
\item Agregar funcionalidad de gestionar vídeos de los pacientes.
\item Agregar funcionalidad de ver vídeos del paciente.
\end{enumerate}
\item \textbf{Periodo}
Este \textsl{sprint} se desarrolló entre el 23 de mayo de 2024 y el 31 de mayo de 2024.
\item \textbf{\textsl{Review}}
Se cumplieron todos los objetivos del sprint.
\end{itemize}

\subsubsection{Sprint 16}
\begin{itemize}
\item \textbf{Objetivos}
\begin{enumerate}
\item Refactorizar la aplicación para aplicar un Modelo-vista-controlador.
\item Solucionar fallos de funcionalidades del doctor a la hora de gestionar vídeos y medicinas.
\item Implementar extracción de características de los vídeos.
\item Realizar comparación de modelos.
\end{enumerate}
\item \textbf{Periodo}
Este \textsl{sprint} se desarrolló entre el 1 de junio de 2024 y el 13 de junio de 2024.
\item \textbf{\textsl{Review}}
Se cumplieron todos los objetivos del sprint.
\end{itemize}
\section{Estudio de viabilidad}
En este apartado se estudian los criterios a tener en cuenta a nivel económico y legal para determinar si es factible y si cumple los objetivos propuestos.

\subsection{Viabilidad económica}
En el siguiente subapartado se expone la viabilidad del proyecto a nivel económico.

\subsubsection{Costes}
Se entiende por coste todo aquel gasto necesario para poder llevar acabo un proyecto desde su inicio hasta su finalización. Para el análisis de los costes del proyecto y el calculo de la amortización se ha tenido en cuenta la duración total del proyecto (40 semanas o entorno a 9 meses).
\begin{enumerate}
\item \textbf{Costes de empleados}: en este caso se cuenta con dos empleados que han sido el desarrollador (el alumno) y el \textit{product owner} (tutor académico).
Se estima que el tiempo real invertido por parte del desarrollador es de entorno a las 550 horas a lo largo de 9 meses. Teniendo en cuenta que el sueldo promedio de un programador junior en España es de entorno a 23.700€\footnote{Fuente:~\url{https://es.indeed.com/career/programador-junior/salaries}} y la jornada laboral completa son aproximadamente 1820 horas, el salario medio bruto por hora es de 13€. Por tanto el salario del desarrollador estará entorno a 550 horas x 13 \(\frac{\text{€}}{\text{hora}}\) = 7150€ o aproximadamente 795\(\frac{\text{€}}{\text{mes}}\)
A esta cantidad se le ha de añadir los impuestos que se han de pagar como empresa por el empleado\footnote{Fuente:~\url{https://www.seg-social.es/wps/portal/wss/internet/Trabajadores/CotizacionRecaudacionTrabajadores/10721/10957/9932/4315}}
\begin{itemize}
\item \textbf{Contingencias Comunes}: correspondiente al 23,60\%.
\item \textbf{Desempleo}: correspondiente al 6,70\%.
\item \textbf{FOGASA}: fondo de garantía salarial. Correspondiente al 0,20\%.
\item \textbf{Formación profesional}: correspondiente al 0,60\%.
\item \textbf{Accidentes de Trabajo y Enfermedades Profesionales}: correspondiente al 1,50\%.
\end{itemize}
Por tanto, tras realizar los cálculos añadiendo los impuestos correspondientes, la empresa debe pagar 1.179,52 \(\frac{\text{€}}{mes}\) por el desarrollador.
\begin{equation}
\frac{795\frac{\text{€}}{mes}}{\text{(1 - (0,236 + 0,067 + 0,002 + 0,006 + 0,015))}} = \text{1.179,52} \frac{\text{€}}{mes}
\end{equation}

Por otro lado el \textit{product owner}, suponiendo un sueldo bruto anual de 39000€\footnote{Fuente:~\url{https://www.glassdoor.es/Sueldos/product-owner-sueldo-SRCH_KO0,13.htm}} , se calcula que el salario bruto hora es de 
21,43€. Teniendo en cuenta que el tutor ha dedicado entorno a 3 horas por semana al proyecto, cobraría 257,16€ brutos al mes. Aplicando la formula al igual que para el desarrollador:
\begin{equation}
\frac{257,16\frac{\text{€}}{mes}}{\text{(1 - (0,236 + 0,067 + 0,002 + 0,006 + 0,015))}} = \text{381,54} \frac{\text{€}}{mes}
\end{equation}
Por lo tanto, y teniendo en cuenta que la duración del proyecto ha sido de 9 meses, el total que ha de pagar la empresa es de 14.049,54€ por los empleados.

\[
381.54\frac{\text{€}}{\text{mes}} \times 9\,\text{meses} + 1179.52\frac{\text{€}}{\text{mes}} \times 9\,\text{meses} = 14049.54\,\text{€}
\]


\item \textbf{Hardware}: para el hardware solo se ha hecho uso del equipo del alumno. Este equipo se trata de un PcCom Gold Élite con un Intel Core i5-11400F, 16 GB de RAM y NVIDIA Geforce RTX 3070 con un P.V.P \footnote{Fuente:~\url{https://www.pccomponentes.com/pccom-gold-elite-intel-core-i5-11400f-16gb-1tb-ssd-rtx-3070}} de 1199,18€. Se calcula que el ordenador será amortizado en 4 años por lo que se debe calcular el precio amortizado para la duración del proyecto. En total son 224,84€.

\[
\frac{1199,18\text{€}\times 9 \text{meses}}{4 \text{años} \times 12 \text{meses}}= 224,85\text{€}
\]

\item \textbf{Software}: la mayoría de programas y \textit{sofware} utilizado durante el desarrollo son gratuitos a excepción de GitHub Copilot cuyo coste de la versión \textit{entreprise} es de 36,35€ al mes. Teniendo en cuenta la duración del proyecto, esto nos da un coste total de 327,15€.

\end{enumerate}

\subsubsection{Costes totales}
A continuación se muestra la tabla con los costes totales.
\begin{table}[ht]
\begin{center}
\begin{tabular}{| c | c |}
	\hline
	\textbf{Concepto} & \textbf{Coste(€)} \\ \hline
	\textbf{Empleados} & 14.049,54 \\
	\textbf{Hardware} & 224,84 \\
	\textbf{Software} & 327,15 \\ \hline
	\hline
	\textbf{Total} & 14.601,53 \\
	\hline
\end{tabular}
\caption{Tabla de costes totales}
\end{center}
\end{table}

\subsubsection{Ingresos}
Dado que este proyecto pretende ser accesible, su uso va a ser completamente gratuito. Debido a esto una de las principales formas en las que se podría obtener rédito económico de esta aplicación sería mediante la implementación de anuncios. Dado que es proyecto pretende ser una herramienta para facilitar el trabajo a pacientes y médicos no se ha considerado oportuno y por tanto no se ha estudiado esta opción. Otra forma en la que se podría sacar ingresos sería mediante donaciones o ayudas que permitiesen cubrir los gastos de mantenimiento y desarrollo del proyecto.

\subsection{Viabilidad legal}

A continuación se muestra una tabla recopilando las licencias de las dependencias utilizadas durante el desarrollo del proyecto.

\begin{longtable}{>{\raggedright\arraybackslash}p{0.3\textwidth} >{\centering\arraybackslash}p{0.15\textwidth} >{\raggedright\arraybackslash}p{0.55\textwidth}}


\toprule
\textbf{Dependencia} & \textbf{Versión} & \textbf{Licencia} \\
\midrule
\endfirsthead

\multicolumn{3}{c}%
{\tablename\ \thetable\ -- \textit{Continuación}} \\
\toprule
\textbf{Dependencia} & \textbf{Versión} & \textbf{Licencia} \\
\midrule
\endhead

\midrule
\multicolumn{3}{r}{\textit{Continúa en la siguiente página}} \\
\endfoot

\bottomrule
\caption[Licencias: dependencias o utilidades]{Dependencias del proyecto y licencias bajo las cuales están registradas. Extraídas mediante la librería pip-licenses\footnotemark .} \\

\endlastfoot

\texttt{Flask} & 3.0.2 & BSD License \\
\texttt{Flask-Login} & 0.6.3 & MIT License \\
\texttt{Flask-SQLAlchemy} & 3.1.1 & BSD License \\
\texttt{Flask-WTF} & 1.2.1 & BSD License \\
\texttt{Font-Awesome-Flask} & 0.1.4 & GNU General Public License v3 or later (GPLv3+) \\
\texttt{Jinja2} & 3.1.3 & BSD License \\
\texttt{MarkupSafe} & 2.1.5 & BSD License \\
\texttt{PyYAML} & 6.0.1 & MIT License \\
\texttt{Pygments} & 2.18.0 & BSD License \\
\texttt{SQLAlchemy} & 2.0.27 & MIT License \\
\texttt{WTForms} & 3.1.2 & BSD License \\
\texttt{Werkzeug} & 3.0.1 & BSD License \\
\texttt{absl-py} & 2.1.0 & Apache Software License \\
\texttt{annotated-types} & 0.6.0 & MIT License \\
\texttt{attrs} & 23.2.0 & MIT License \\
\texttt{blinker} & 1.7.0 & MIT License \\
\texttt{certifi} & 2024.2.2 & Mozilla Public License 2.0 (MPL 2.0) \\
\texttt{cffi} & 1.16.0 & MIT License \\
\texttt{charset-normalizer} & 3.3.2 & MIT License \\
\texttt{click} & 8.1.7 & BSD License \\
\texttt{cloudpickle} & 3.0.0 & BSD License \\
\texttt{colorama} & 0.4.6 & BSD License \\
\texttt{contourpy} & 1.2.1 & BSD License \\
\texttt{cycler} & 0.12.1 & BSD License \\
\texttt{dask} & 2024.5.0 & BSD License \\
\texttt{dask-expr} & 1.1.0 & BSD License \\
\texttt{distributed} & 2024.5.0 & BSD License \\
\texttt{flatbuffers} & 24.3.25 & Apache Software License \\
\texttt{fonttools} & 4.51.0 & MIT License \\
\texttt{fsspec} & 2024.3.1 & BSD License \\
\texttt{greenlet} & 3.0.3 & MIT License \\
\texttt{gunicorn} & 22.0.0 & MIT License \\
\texttt{idna} & 3.7 & BSD License \\
\texttt{importlib\_metadata} & 7.1.0 & Apache Software License \\
\texttt{itsdangerous} & 2.1.2 & BSD License \\
\texttt{jax} & 0.4.27 & Apache-2.0 \\
\texttt{jaxlib} & 0.4.27 & Apache-2.0 \\
\texttt{joblib} & 1.4.2 & BSD License \\
\texttt{kiwisolver} & 1.4.5 & BSD License \\
\texttt{llvmlite} & 0.42.0 & BSD \\
\texttt{locket} & 1.0.0 & BSD License \\
\texttt{matplotlib} & 3.8.4 & Python Software Foundation License \\
\texttt{mediapipe} & 0.10.13 & Apache Software License \\
\texttt{ml-dtypes} & 0.4.0 & Apache Software License \\
\texttt{msgpack} & 1.0.8 & Apache Software License \\
\texttt{numba} & 0.59.1 & BSD License \\
\texttt{numpy} & 1.26.4 & BSD License \\
\texttt{opencv-contrib-python} & 4.9.0.80 & Apache Software License \\
\texttt{opencv-python} & 4.9.0.80 & Apache Software License \\
\texttt{opt-einsum} & 3.3.0 & MIT \\
\texttt{packaging} & 24.0 & Apache Software License; BSD License \\
\texttt{pandas} & 2.2.2 & BSD License \\
\texttt{partd} & 1.4.2 & BSD \\
\texttt{patsy} & 0.5.6 & BSD License \\
\texttt{pillow} & 10.3.0 & Historical Permission Notice and Disclaimer (HPND) \\
\texttt{pip-licenses} & 4.4.0 & MIT License \\
\texttt{protobuf} & 4.25.3 & 3-Clause BSD License \\
\texttt{psutil} & 5.9.8 & BSD License \\
\texttt{pure-eval} & 0.2.2 & MIT License \\
\texttt{pyarrow} & 16.0.0 & Apache Software License \\
\texttt{pycparser} & 2.22 & BSD License \\
\texttt{pydantic} & 2.7.1 & MIT License \\
\texttt{pydantic-settings} & 2.2.1 & MIT License \\
\texttt{pydantic\_core} & 2.18.2 & MIT License \\
\texttt{pyparsing} & 3.1.2 & MIT License \\
\texttt{python-dateutil} & 2.9.0.post0 & Apache Software License; BSD License \\
\texttt{python-dotenv} & 1.0.1 & BSD License \\
\texttt{pytz} & 2024.1 & MIT License \\
\texttt{pywin32} & 306 & Python Software Foundation License \\
\texttt{pyzmq} & 26.0.3 & BSD License \\
\texttt{requests} & 2.31.0 & Apache Software License \\
\texttt{scikit-learn} & 1.4.2 & BSD License \\
\texttt{scipy} & 1.13.0 & BSD License \\
\texttt{six} & 1.16.0 & MIT License \\
\texttt{sortedcontainers} & 2.4.0 & Apache Software License \\
\texttt{sounddevice} & 0.4.6 & MIT License \\
\texttt{statsmodels} & 0.14.2 & BSD License \\
\texttt{stumpy} & 1.12.0 & BSD License \\
\texttt{tblib} & 3.0.0 & BSD License \\
\texttt{threadpoolctl} & 3.5.0 & BSD License \\
\texttt{toolz} & 0.12.1 & BSD License \\
\texttt{tornado} & 6.4 & Apache Software License \\
\texttt{tqdm} & 4.66.4 & MIT License; Mozilla Public License 2.0 (MPL 2.0) \\
\texttt{traitlets} & 5.14.3 & BSD License \\
\texttt{tsfresh} & 0.20.2 & MIT \\
\texttt{typing\_extensions} & 4.9.0 & Python Software Foundation License \\
\texttt{tzdata} & 2024.1 & Apache Software License \\
\texttt{urllib3} & 2.2.1 & MIT License \\
\texttt{zict} & 3.0.0 & BSD License \\
\texttt{zipp} & 3.18.1 & MIT License \\

\end{longtable}

\footnotetext{\url{https://pypi.org/project/pip-licenses/}}

Todas las librerías, dependencias y herramientas utilizadas licencias encontradas en programas de código abierto. Además, el Trabajo de Fin de Grado PADDEL~\cite{paddelRepo} que ha sido utilizado para el desarrollo de este proyecto tiene una licencia MIT la cual permite a cualquier persona que obtenga una copia del proyecto modificarlo, copiarlo y redistribuirlo libremente. Es por ello que se ha optado por utilizar la licencia MIT para este proyecto también. 



