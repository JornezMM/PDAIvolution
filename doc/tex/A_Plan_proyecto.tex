\apendice{Plan de Proyecto Software}

\section{Introducción}
El propósito de este documento es establecer una guía de forma clara y organizado para la ejecución de un proyecto. En el se definen los objetivos y la forma en la que se van a alcanzar.


\subsection{Scrum}
Para la planificación del proyecto ha sido empleada la metodología ágil Scrum \cite{scrumMaster2022}. Scrum es una metodología ágil para la gestión de proyectos complejos en entornos cambiantes. Se basa en tres pilares: eventos, roles y artefactos, y se trabaja en sprints de una duración determinada, generalmente entre una semana y un mes. Los equipos de Scrum se autoorganizan y se comprometen a entregar resultados de alta calidad de manera eficiente y creativa.
\subsubsection{Roles}
En la metodología Scrum, se destacan tres roles principales:

\begin{itemize}
\item Scrum \textsl{Master}: Líder del equipo encargado de eliminar obstáculos y facilitar la autoorganización y coordinación del equipo.
\item \textsl{Product Owner}: Representa la voz del cliente, lidera el desarrollo del producto y busca el valor para los usuarios.
\item Equipo de desarrollo: Responsable de ejecutar las acciones previstas para el éxito del proceso
\end{itemize}
\subsubsection{Eventos}
Los eventos clave en Scrum son:
\begin{itemize}


\item \textsl{Sprint}: Reunión para planificar el trabajo del próximo sprint.
\item Scrum diario: Breve reunión diaria para sincronizar actividades.
\item Revisión del \textsl{sprint}: Revisión del producto al final del \textsl{sprint}.
\item Retrospectiva del \textsl{sprint} : Análisis de los éxitos y fallos del \textsl{sprint}
\end{itemize}




\subsubsection{Artefactos}
Scrum utiliza tres artefactos principales:
\begin{itemize}
\item Pila de producto o \textsl{product backlog}: Inventario que contiene el trabajo pendiente en el producto.
\item Pila del \textsl{sprint} o \textsl{sprint backlog}: Lista de tareas a realizar durante el \textsl{sprint}.
\item Incremento: Versión mejorada y funcional del producto al final de cada \textsl{sprint}.
\end{itemize}




 
\section{Planificación temporal}
\subsection{Planificación mediante \textsl{sprints}}
Se ha decido que la planificación del proyecto se haga mediante \textsl{sprints} debido al tamaño del equipo (un desarrollador).


\subsubsection{Sprint 1}
\begin{itemize}
\item \textbf{Objetivos}
\begin{enumerate}
\item Configuración inicial: creación del repositorio y configuración de cuenta de ZenHub
\item Memoria: redacción de la introducción y familarización con las aplicaciones para edición de archivos tex
\item Lectura de papers:  Supervised classification of bradykinesia in Parkinson’s disease from smartphone videos  \cite{williams2020discerning}, The discerning eye of computer vision: Can it measure Parkinson's finger tap bradykinesia? \cite{williams2020supervised}  y A computer vision framework for finger-tapping evaluation in Parkinson's disease \cite{khan2014computer}.
\end{enumerate}
\item \textbf{Periodo}
Este \textsl{sprint} se desarrolló entre el 2 de octubre del 2023 y el 16 de octubre del 2023.
\item \textbf{\textsl{Review}}
En la revisión se resolvieron dudas sobre la documentación y se decidio cambiar de herramienta para la realización de la metodología scrum del proyecto a zube. 


\end{itemize}

\subsubsection{Sprint 2}
\begin{itemize}
\item \textbf{Objetivos}
\begin{enumerate}
\item Documentación de trabajos previos: lectura del trabajo realizado por Catalin para ver la extracción de datos.
\item Memoria: redacción del apartado de trabajos relacionados.
\item Curso de flask: realización de los tutoriales del framework de flask para la realización del apartado web.
\end{enumerate}
\item \textbf{Periodo}
Este \textsl{sprint} se desarrolló entre el 16 de octubre del 2023 y el 29 de octubre del 2023.
\item \textbf{\textsl{Review}}
Se resolvieron dudas sobre donde encontrar la información dentro del proyecto de Catalin así como problemas a la hora de compilar los documentos de latex. 


\end{itemize}
\subsubsection{Sprint 3}
\begin{itemize}
\item \textbf{Objetivos}
\begin{enumerate}
\item Documentación de trabajos previos: lectura de los documentos para la extracción de datos usando el paquete de python TSFresh.
\item Memoria: corrección de errores en la documentación y añadida biografía.
\item Curso de flask: continuada la realización de los tutoriales del framework de flask para la realización del apartado web.
\item Repositorio: añadido el archivo gitignore para evitar la subida de archivos temporales.
\end{enumerate}
\item \textbf{Periodo}
Este \textsl{sprint} se desarrolló entre el 30 de octubre del 2023 y el 12 de noviembre del 2023.
\item \textbf{\textsl{Review}}
Problemas a la hora de la realización de los objetivos propuestos agregados en el siguiente \textsl{sprint}. 


\end{itemize}


\section{Estudio de viabilidad}

\subsection{Viabilidad económica}

\subsection{Viabilidad legal}


