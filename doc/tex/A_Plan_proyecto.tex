\apendice{Plan de Proyecto Software}

\section{Introducción}
El propósito de este documento es establecer una guía de forma clara y organizada para la ejecución de un proyecto. En él se definen los objetivos y la forma en la que se van a alcanzar.


\subsection{Scrum}
Para la planificación del proyecto ha sido empleada la metodología ágil Scrum \cite{scrumMaster2022}. Scrum es una metodología ágil para la gestión de proyectos complejos en entornos cambiantes. Se basa en tres pilares: eventos, roles y artefactos, y se trabaja en sprints de una duración determinada, generalmente entre una semana y un mes. Los equipos de Scrum se autoorganizan y se comprometen a entregar resultados de alta calidad de manera eficiente y creativa.
\subsubsection{Roles}
En la metodología Scrum, se destacan tres roles principales:

\begin{itemize}
\item \textsl{Scrum Master}: Líder del equipo encargado de eliminar obstáculos y facilitar la auto-organización y coordinación del equipo.
\item \textsl{Product Owner}: Representa la voz del cliente, lidera el desarrollo del producto y busca el valor para los usuarios.
\item Equipo de desarrollo: Responsable de ejecutar las acciones previstas para el éxito del proceso
\end{itemize}
\subsubsection{Eventos}
Los eventos clave en Scrum son:
\begin{itemize}


\item \textsl{Sprint}: Reunión para planificar el trabajo del próximo sprint.
\item Scrum diario: Breve reunión diaria para sincronizar actividades.
\item Revisión del \textsl{sprint}: Revisión del producto al final del \textsl{sprint}.
\item Retrospectiva del \textsl{sprint} : Análisis de los éxitos y fallos del \textsl{sprint}
\end{itemize}




\subsubsection{Artefactos}
Scrum utiliza tres artefactos principales:
\begin{itemize}
\item Pila de producto o \textsl{product backlog}: Inventario que contiene el trabajo pendiente en el producto.
\item Pila del \textsl{sprint} o \textsl{sprint backlog}: Lista de tareas a realizar durante el \textsl{sprint}.
\item Incremento: Versión mejorada y funcional del producto al final de cada \textsl{sprint}.
\end{itemize}




 
\section{Planificación temporal}
\subsection{Planificación mediante \textsl{sprints}}
Se ha decido que la planificación del proyecto se haga mediante \textsl{sprints} debido al tamaño del equipo (un desarrollador).


\subsubsection{Sprint 1}
\begin{itemize}
\item \textbf{Objetivos}
\begin{enumerate}
\item Configuración inicial: creación del repositorio y configuración de cuenta de ZenHub.
\item Memoria: redacción de la introducción y familarización con las aplicaciones para edición de archivos \TeX .
\item Lectura de papers:  <<Supervised classification of bradykinesia in Parkinson’s disease from smartphone videos>>  \cite{williams2020discerning}, <<The discerning eye of computer vision: Can it measure Parkinson's finger tap bradykinesia?>> \cite{williams2020supervised}  y <<A computer vision framework for finger-tapping evaluation in Parkinson's disease>> \cite{khan2014computer}.
\end{enumerate}
\item \textbf{Periodo}
Este \textsl{sprint} se desarrolló entre el 2 de octubre del 2023 y el 16 de octubre del 2023.
\item \textbf{\textsl{Review}}
En la revisión se resolvieron dudas sobre la documentación y se decidió cambiar de herramienta para la realización de la metodología scrum del proyecto a zube. 


\end{itemize}

\subsubsection{Sprint 2}
\begin{itemize}
\item \textbf{Objetivos}
\begin{enumerate}
\item Documentación de trabajos previos: lectura del trabajo realizado por Catalin para ver la extracción de datos.
\item Memoria: redacción del apartado de trabajos relacionados.
\item Curso de flask: realización de los tutoriales del framework de flask para la realización del apartado web.
\end{enumerate}
\item \textbf{Periodo}
Este \textsl{sprint} se desarrolló entre el 16 de octubre del 2023 y el 29 de octubre del 2023.
\item \textbf{\textsl{Review}}
Se resolvieron dudas sobre dónde encontrar la información dentro del proyecto de Catalin así como problemas a la hora de compilar los documentos de \LaTeX . 


\end{itemize}
\subsubsection{Sprint 3}
\begin{itemize}
\item \textbf{Objetivos}
\begin{enumerate}
\item Documentación de trabajos previos: lectura de los documentos para la extracción de datos usando el paquete de python TSFresh.
\item Memoria: corrección de errores en la documentación y añadida biografía.
\item Curso de flask: continuada la realización de los tutoriales del framework de flask para la realización del apartado web.
\item Repositorio: añadido el archivo gitignore para evitar la subida de archivos temporales.
\end{enumerate}
\item \textbf{Periodo}
Este \textsl{sprint} se desarrolló entre el 30 de octubre del 2023 y el 12 de noviembre del 2023.
\item \textbf{\textsl{Review}}
Problemas a la hora de la realización de los objetivos propuestos agregados en el siguiente \textsl{sprint}. 


\end{itemize}

\subsubsection{Sprint 4}
\begin{itemize}
\item \textbf{Objetivos}
\begin{enumerate}
\item Creación del \textit{mockup} de la aplicación.
\item Documentación de los anexos.
\end{enumerate}
\item \textbf{Periodo}
Este \textsl{sprint} se desarrolló entre el 13 de noviembre del 2023 y el 30 de diciembre del 2023.
\item \textbf{\textsl{Review}}
Se revisó el \textit{mockup} y se propuso una modificación del mismo. 


\end{itemize}


\subsubsection{Sprint 5}
\begin{itemize}
\item \textbf{Objetivos}
\begin{enumerate}
\item Finalización del curso de flask.
\item Modificación del \textit{mockup}.
\item Creación de diagramas entidad relación y diagramas de casos de uso.
\end{enumerate}
\item \textbf{Periodo}
Este \textsl{sprint} se desarrolló entre el 30 de noviembre del 2023 y el 12 de diciembre del 2023.
\item \textbf{\textsl{Review}}
Se revisaron tanto los diagramas como el \textit{mockup} y se propusieron unos cambios en los diagramas. 


\end{itemize}

\subsubsection{Sprint 6}
\begin{itemize}
\item \textbf{Objetivos}
\begin{enumerate}
\item Documentación general añadiendo técnicas y herramientas.
\item Búsqueda de trabajos parecidos o aplicaciones médicas similares.
\item Modificación de los diagramas de casos de uso y entidad-relación.
\end{enumerate}
\item \textbf{Periodo}
Este \textsl{sprint} se desarrolló entre el 13 de diciembre del 2023 y el 19 de enero de 2024.
\item \textbf{\textsl{Review}}
Se cumplieron todos los objetivos del sprint. 


\end{itemize}
\subsubsection{Sprint 7}
\begin{itemize}
\item \textbf{Objetivos}
\begin{enumerate}
\item Documentación general añadiendo técnicas y herramientas.
\item Búsqueda de trabajos parecidos o aplicaciones médicas similares.
\item Modificación del diagramas de casos de uso y entidad-relación.
\end{enumerate}
\item \textbf{Periodo}
Este \textsl{sprint} se desarrolló entre el 20 de enero de 2024 y el 15 de febrero de 2024.
\item \textbf{\textsl{Review}}
Se cumplieron todos los objetivos del sprint. Se detectaron ciertos fallos en la documentación que se añadieron como tarea para el siguiente sprint. 


\end{itemize}
\subsubsection{Sprint 8}
\begin{itemize}
\item \textbf{Objetivos}
\begin{enumerate}
\item Comienzo de la página web.
\item Conceptos teóricos de la aplicación.
\item Solución de errores detectados en la anterior revisión.
\item Conexión con la base de datos.
\end{enumerate}
\item \textbf{Periodo}
Este \textsl{sprint} se desarrolló entre el 16 de febrero del 2024 y el 29 de febrero de 2024.
\item \textbf{\textsl{Review}}
Se cumplieron todos los objetivos del sprint. 
\end{itemize}

\subsubsection{Sprint 9}
\begin{itemize}
\item \textbf{Objetivos}
\begin{enumerate}
\item Continuación de la redacción de los conceptos teóricos.
\item Creación de la página de registro.
\end{enumerate}
\item \textbf{Periodo}
Este \textsl{sprint} se desarrolló entre el 1 de marzo del 2024 y el 14 de marzo de 2024.
\item \textbf{\textsl{Review}}
No se cumplió el objetivo de la documentación que se añadió al siguiente sprint. 
\end{itemize}

\subsubsection{Sprint 10}
\begin{itemize}
\item \textbf{Objetivos}
\begin{enumerate}
\item Continuación de la redacción de los conceptos teóricos.
\item Creación de la página de inicio de sesión.
\end{enumerate}
\item \textbf{Periodo}
Este \textsl{sprint} se desarrolló entre el 15 de marzo del 2024 y el 8 de abril de 2024.
\item \textbf{\textsl{Review}}
Se cumplieron todos los objetivos del sprint.  
\end{itemize}

\subsubsection{Sprint 11}
\begin{itemize}
\item \textbf{Objetivos}
\begin{enumerate}
\item Documentación de técnicas y herramientas.
\item Creación de la página principal de administrador.
\item Solución de errores relacionados con la página de registro.
\item Creación de la página de modificación de usuarios.
\end{enumerate}
\item \textbf{Periodo}
Este \textsl{sprint} se desarrolló entre el 9 de abril de 2024 y el 25 de abril de 2024.
\item \textbf{\textsl{Review}}
Se cumplieron todos los objetivos del sprint. 
\end{itemize}

\subsubsection{Sprint 12}
\begin{itemize}
\item \textbf{Objetivos}
\begin{enumerate}
\item Creación de la página principal del paciente.
\item Añadir la funcionalidad de actualizar el usuario en la base de datos una vez modificado.
\item Iniciar entrenamiento de los modelos para la predicción de características.
\end{enumerate}
\item \textbf{Periodo}
Este \textsl{sprint} se desarrolló entre el 26 de abril de 2024 y el 8 de mayo de 2024.
\item \textbf{\textsl{Review}}
No se pudieron realizar los experimentos debido a problemas de compatibilidad de versión. 
\end{itemize}

\subsubsection{Sprint 13}
\begin{itemize}
\item \textbf{Objetivos}
\begin{enumerate}
\item Creación de la página principal del doctor.
\item Continuar con el entrenamiento de los modelos.
\end{enumerate}
\item \textbf{Periodo}
Este \textsl{sprint} se desarrolló entre el 9 de mayo de 2024 y el 15 de mayo de 2024.
\item \textbf{\textsl{Review}}
Debido a problemas durante el entrenamiento de los modelos, no se pudo completar este objetivo correctamente. 
\end{itemize}

\subsubsection{Sprint 14}
\begin{itemize}
\item \textbf{Objetivos}
\begin{enumerate}
\item Continuar con el entrenamiento de los modelos.
\end{enumerate}
\item \textbf{Periodo}
Este \textsl{sprint} se desarrolló entre el 16 de mayo de 2024 y el 22 de mayo de 2024.
\item \textbf{\textsl{Review}}
Se cumplieron todos los objetivos del sprint y se consiguió resolver el problema con el entrenamiento de modelos.
\end{itemize}

\subsubsection{Sprint 15}
\begin{itemize}
\item \textbf{Objetivos}
\begin{enumerate}
\item Agregar funcionalidad de gestionar medicinas de los pacientes.
\item Agregar funcionalidad de ver medicinas del paciente.
\item Modificar la base de datos para añadir medicinas a la aplicación.
\item Agregar funcionalidad de gestionar vídeos de los pacientes.
\item Agregar funcionalidad de ver vídeos del paciente.
\end{enumerate}
\item \textbf{Periodo}
Este \textsl{sprint} se desarrolló entre el 23 de mayo de 2024 y el 31 de mayo de 2024.
\item \textbf{\textsl{Review}}
Se cumplieron todos los objetivos del sprint.
\end{itemize}

\subsubsection{Sprint 16}
\begin{itemize}
\item \textbf{Objetivos}
\begin{enumerate}
\item Refactorizar la aplicación para aplicar un Modelo-vista-controlador.
\item Solucionar fallos de funcionalidades del doctor a la hora de gestionar vídeos y medicinas.
\item Implementar extracción de características de los vídeos.
\item Realizar comparación de modelos.
\end{enumerate}
\item \textbf{Periodo}
Este \textsl{sprint} se desarrolló entre el 1 de junio de 2024 y el 13 de junio de 2024.
\item \textbf{\textsl{Review}}
Se cumplieron todos los objetivos del sprint.
\end{itemize}
\section{Estudio de viabilidad}

\subsection{Viabilidad económica}

\subsection{Viabilidad legal}


