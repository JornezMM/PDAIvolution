\capitulo{1}{Introducción}

El parkinson es una enfermedad crónica e irreversible que afecta a más de 150.000 personas en toda España, siendo esta la segunda enfermedad neurodegenerativa más común después del Alzheimer ~\cite{brainsci11081027}. Esta enfermedad se caracteriza por la perdida de neuronas dopaminergicas lo que conlleva un desequilibrio en los circuitos neuronales que controlan el movimiento esto puede provocar temblores, rigidez muscular, lentitud de movimiento (bradicenesia) y problemas de equilibrio y coordinación~\cite{Poewe_Seppi_Tanner_Halliday_Brundin_Volkmann_Schrag_Lang_2017}.
Pese a que no exista una cura, saber la evolución del paciente y su respuesta a los medicamentos y tratamientos puede ayudar enormemente tanto a pacientes como a médicos a adecuarse mejor a las circunstancias del paciente y con ello mejorar su calidad de vida.
 
La bradicenesia es uno de los síntomas más importantes a la hora de clasificar el grado de la enfermedad. Este síntoma se hace muy visible en un ejercicio conocido como test de golpeteo rápido de los dedos (o <<rapid finger tapping test>> en inglés). Este ejercicio consiste en separar y juntar repetidas veces los dedos indice y pulgar de forma rápida y constante.

En las últimas décadas, el campo de la inteligencia artificial ha experimentado un gran avance, con aplicaciones prometedoras en el ámbito de la salud. Específicamente, las técnicas de aprendizaje automático han demostrado su potencial para el diagnóstico y clasificación de enfermedades neurodegenerativas a partir de datos clínicos, genéticos y de neuroimagen ~\cite{CardiacCare}.

En este proyecto se ha hecho uso de la inteligencia artificial para la creación de modelos capaces de clasificar el grado de bradicinesia en vídeos del <<rapid finger tapping test>>. Ver de manera individual la clasificación de cada vídeo puede no ser del todo útil para poder comprobar la evolución del paciente. Es por ello que se ha decido crear, usando vídeos de distintos días, una gráfica en la que se pueden ver las dos características de la bradicenesia: la amplitud y la velocidad. 