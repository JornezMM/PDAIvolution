\newcommand{\cu}[9]{
    \begin{table}[p]
        \centering
        \begin{tabularx}{\linewidth}{ p{0.21\columnwidth} p{0.71\columnwidth} }
            \toprule
            \textbf{CU-#1} & \textbf{#2} \\
            \toprule
            \textbf{Versión} & 1.0 \\
            \textbf{Autor} & \theauthor \\
            \textbf{Requisitos asociados} & #3 \\
            \textbf{Descripción} & #4 \\
            \textbf{Precondición} & #5 \\
            \textbf{Acciones} & #6 \\
            \textbf{Postcondición} & #7 \\
            \textbf{Excepciones} & #8 \\
			\textbf{Importancia} & #9 \\
            \bottomrule
        \end{tabularx}
        \caption{CU-#1 #2.}
    \end{table}
}

\apendice{Especificación de Requisitos}

\section{Introducción}
Previo a la creación de un programa se debe especificar las características concretas que ha de tener.
En este caso caso ha definido estas características utilizando UML como estándar.



\section{Objetivos generales}
El objetivo principal de esta aplicación es mostrar tanto a los pacientes como a sus respectivos médicos ver su evolución mediante el uso de inteligencia artificial.

Además se busca que el usuario médico sea capaz de realizar comparaciones de múltiples pacientes de manera simultanea.

\section{Catálogo de requisitos}
\subsection{Requisitos funcionales}
\begin{enumerate}[label=\textbf{RF\arabic*:}, left=0pt, itemindent=1.5em]

\item El sistema debe permitir diferenciar entre tres tipos de roles: administrador, paciente y médico.
\item El usuario con rol de paciente ha de ser capaz de agregar vídeos.
\item El usuario con rol de paciente ha de ser capaz de eliminar sus vídeos.
\item El usuario con rol de paciente ha de ser capaz de modificar sus vídeos.
\item El paciente y el médico han de ser capaces de clasificar los vídeos.
\item Los usuarios han de ser capaces de iniciar sesión con sus correspondientes credenciales.
\item Los usuarios deben ser capaces de poder finalizar sesión.
\item El administrador ha de ser capaz de añadir usuarios.
\item El administrador ha de ser capaz de eliminar usuarios.
\item El administrador ha de ser capaz de modificar usuarios.
\item El administrador ha de ser capaz de visualizar a los usuarios.
\item Los médicos han de ser capaces de agregarse pacientes asociados.
\item Los médicos han de ser capaces de eliminar sus pacientes asociados.
\item El paciente y el médico han de ser capaces de ver la evolución del paciente.

\end{enumerate}

\section{Especificación de requisitos}

\cu{1}{Iniciar sesión}
{RF-1,RF-6}
{Los usuarios deben ser capaces de iniciar sesión}
{	\begin{itemize}
	\def\labelenumi{\arabic{enumi}.}
	\tightlist
	\item El usuario debe estar dado de alta en la base de datos.
	\item Estar en la página de inicio de sesión.
	\end{itemize}}
{
	\begin{itemize}
	\def\labelenumi{\arabic{enumi}.}
	\tightlist

    \item El usuario introduce su nombre de usuario.
    \item El usuario introduce su contraseña.
    \item El usuario hace clic en el botón de iniciar sesión.
    \end{itemize}
}
{El usuario será redirijo a su página de perfil correspondiente}
{	\begin{itemize}
	\def\labelenumi{\arabic{enumi}.}
	\tightlist

    \item Las credenciales no son correctas.
    \item El usuario no está dado de alta.
    \end{itemize}}
{Alta}

\cu{2}{Finalizar sesión}
{RF-1,RF-7}
{Los usuarios deben ser capaces de iniciar sesión}
{	\begin{itemize}
	\def\labelenumi{\arabic{enumi}.}
	\tightlist
	\item El usuario debe tener la sesión iniciada.
	\end{itemize}}
{
	\begin{itemize}
	\def\labelenumi{\arabic{enumi}.}
	\tightlist

    \item El usuario hace clic en el botón de finalizar sesión.
    \end{itemize}
}
{El usuario será redirijo a su página de perfil correspondiente}
{}
{Alta}

\cu{3}{Visualizar evolución}
{RF-14}
{El paciente y el médico han de ser capaces de visualizar la evolución del paciente}
{	\begin{itemize}
	\def\labelenumi{\arabic{enumi}.}
	\tightlist
	\item El paciente debe estar dado de alta en la base de datos.
	\item El paciente debe de tener al menos 2 vídeos para ver la gráfica.
	\end{itemize}}
{
	\begin{itemize}
	\def\labelenumi{\arabic{enumi}.}
	\tightlist

    \item El usuario accede a su perfil.
    \item Se visualiza la gráfica de la evolución del paciente.
    \end{itemize}
}
{Se muestra al usuario su gráfico correspondiente}
{}
{Alta}

\cu{4.1}{Dar de alta a usuarios}
{RF-8}
{El administrador debe poder dar de alta a nuevos usuarios.}
{	\begin{itemize}
	\def\labelenumi{\arabic{enumi}.}
	\tightlist
	\item El usuario no debe estar dado de alta.
	\end{itemize}}
{
	\begin{itemize}
	\def\labelenumi{\arabic{enumi}.}
	\tightlist

    \item El administrador pulsa el botón de <<añadir usuario>>.
    \item El administrador introduce los datos del paciente.
    \item El administrador confirma los datos y pulsa el botón <<añadir>>.
    \end{itemize}
}
{Se muestra el nuevo usuario en la lista de usuarios}
{\begin{itemize}
	\def\labelenumi{\arabic{enumi}.}
	\tightlist

    \item El usuario ya existe.
    \item Los datos introducidos son erróneos.
    \end{itemize}}
{Alta}

\cu{4.2}{Dar de baja a usuarios}
{RF-9}
{El administrador debe poder dar de baja a usuarios.}
{	\begin{itemize}
	\def\labelenumi{\arabic{enumi}.}
	\tightlist
	\item El usuario debe estar dado de alta.
	\end{itemize}}
{
	\begin{itemize}
	\def\labelenumi{\arabic{enumi}.}
	\tightlist

    \item El administrador pulsa el botón de <<eliminar usuario>>.
    \item El administrador confirma el usuario y pulsa el botón <<eliminar>>.
    \end{itemize}
}
{Ya no se muestra al usuario en la lista de usuarios}
{\begin{itemize}
	\def\labelenumi{\arabic{enumi}.}
	\tightlist

    \item El usuario no existe.
    \end{itemize}}
{Alta}

\cu{4.3}{Modificar usuarios}
{RF-10}
{El administrador debe poder modificar usuarios.}
{	\begin{itemize}
	\def\labelenumi{\arabic{enumi}.}
	\tightlist
	\item El usuario debe estar dado de alta.
	\end{itemize}}
{
	\begin{itemize}
	\def\labelenumi{\arabic{enumi}.}
	\tightlist

    \item El administrador pulsa el botón de <<modificar usuario>>.
    \item El administrador realiza los cambios del usuario y pulsa el botón <<confirmar>>.
    \end{itemize}
}
{Se muestra la modificación del usuario.}
{\begin{itemize}
	\def\labelenumi{\arabic{enumi}.}
	\tightlist

    \item El usuario modificado coincide con otro que ya existe.
    \end{itemize}}
{Alta}

\cu{4.4}{Visualizar usuarios}
{RF-11}
{El administrador debe poder visualizar la vista de usuarios.}
{	\begin{itemize}
	\def\labelenumi{\arabic{enumi}.}
	\tightlist
	\item Debe existir al menos 1 usuario.
	\end{itemize}}
{
	\begin{itemize}
	\def\labelenumi{\arabic{enumi}.}
	\tightlist

    \item El administrador visualiza la lista de usuarios.
    \end{itemize}
}
{Se muestra la lista de usuarios.}
{}
{Media}