\apendice{Especificación de Requisitos}

\section{Introducción}
Previo a la creación de un programa se debe especificar las características concretas que ha de tener.
En este caso caso ha definido estas características utilizando UML como estándar.



\section{Objetivos generales}
El objetivo principal de esta aplicación es mostrar tanto a los pacientes como a sus respectivos médicos ver su evolución mediante el uso de inteligencia artificial.

Además se busca que el usuario médico sea capaz de realizar comparaciones de múltiples pacientes de manera simultanea.

\section{Catálogo de requisitos}
\subsection{Requisitos funcionales}
\begin{enumerate}[label=\textbf{RF\arabic*:}, left=0pt, itemindent=1.5em]

\item El sistema debe permitir diferenciar entre tres tipos de roles: administrador, paciente y médico.
\item El usuario con rol de paciente ha de ser capaz de agregar vídeos.
\item El usuario con rol de paciente ha de ser capaz de eliminar sus vídeos.
\item El usuario con rol de paciente ha de ser capaz de modificar sus vídeos.
\item El paciente y el médico han de ser capaces de clasificar los vídeos.
\item Los usuarios han de ser capaces de iniciar sesión con sus correspondientes credenciales.
\item Los usuarios deben ser capaces de poder finalizar sesión.
\item El administrador ha de ser capaz de añadir usuarios.
\item El administrador ha de ser capaz de eliminar usuarios.
\item El administrador ha de ser capaz de modificar usuarios.
\item El administrador ha de ser capaz de visualizar a los usuarios.
\item Los médicos han de ser capaces de agregarse pacientes asociados.
\item Los médicos han de ser capaces de eliminar sus pacientes asociados.

\end{enumerate}

\section{Especificación de requisitos}


