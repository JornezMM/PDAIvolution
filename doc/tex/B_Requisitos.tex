\newcommand{\cu}[9]{
    \begin{table}[p]
        \centering
        \begin{tabularx}{\linewidth}{ p{0.21\columnwidth} p{0.71\columnwidth} }
            \toprule
            \textbf{CU-#1} & \textbf{#2} \\
            \toprule
            \textbf{Versión} & 1.0 \\
            \textbf{Autor} & \theauthor \\
            \textbf{Requisitos asociados} & #3 \\
            \textbf{Descripción} & #4 \\
            \textbf{Precondición} & #5 \\
            \textbf{Acciones} & #6 \\
            \textbf{Postcondición} & #7 \\
            \textbf{Excepciones} & #8 \\
			\textbf{Importancia} & #9 \\
            \bottomrule
        \end{tabularx}
        \caption{CU-#1 #2.}
    \end{table}
}

\apendice{Especificación de Requisitos}

\section{Introducción}
Previo a la creación de un programa se debe especificar las características concretas que ha de tener.
En este caso caso ha definido estas características utilizando UML como estándar.



\section{Objetivos generales}
El objetivo principal de esta aplicación es mostrar tanto a los pacientes como a sus respectivos médicos ver su evolución mediante el uso de inteligencia artificial.

Además se busca que el usuario médico sea capaz de realizar comparaciones de múltiples pacientes de manera simultanea.

\section{Catálogo de requisitos}
\subsection{Requisitos funcionales}
\begin{enumerate}[label=\textbf{RF\arabic*:}, left=0pt, itemindent=1.5em]

\item El sistema debe permitir diferenciar entre tres tipos de roles: administrador, paciente y médico.
\item El usuario con rol de paciente ha de ser capaz de agregar vídeos.
\item El usuario con rol de paciente ha de ser capaz de eliminar sus vídeos.
\item El usuario con rol de paciente ha de ser capaz de modificar sus vídeos.
\item El paciente y el médico han de ser capaces de clasificar los vídeos.
\item Los usuarios han de ser capaces de iniciar sesión con sus correspondientes credenciales.
\item Los usuarios deben ser capaces de poder finalizar sesión.
\item El administrador ha de ser capaz de añadir usuarios.
\item El administrador ha de ser capaz de eliminar usuarios.
\item El administrador ha de ser capaz de modificar usuarios.
\item El administrador ha de ser capaz de visualizar a los usuarios.
\item Los médicos han de ser capaces de agregarse pacientes asociados.
\item Los médicos han de ser capaces de eliminar sus pacientes asociados.
\item El paciente y el médico han de ser capaces de ver la evolución del paciente.

\end{enumerate}

\section{Especificación de requisitos}

\cu{1}{Iniciar sesión}
{RF-1,RF-6}
{Los usuarios deben ser capaces de iniciar sesión}
{	\begin{itemize}
	\def\labelenumi{\arabic{enumi}.}
	\tightlist
	\item El usuario debe estar dado de alta en la base de datos.
	\item Estar en la página de inicio de sesión.
	\end{itemize}}
{
	\begin{itemize}
	\def\labelenumi{\arabic{enumi}.}
	\tightlist

    \item El usuario introduce su nombre de usuario.
    \item El usuario introduce su contraseña.
    \item El usuario hace clic en el botón de iniciar sesión.
    \end{itemize}
}
{El usuario será redirijo a su página de perfil correspondiente}
{	\begin{itemize}
	\def\labelenumi{\arabic{enumi}.}
	\tightlist

    \item Las credenciales no son correctas.
    \item El usuario no está dado de alta.
    \end{itemize}}
{Alta}

\cu{2}{Finalizar sesión}
{RF-1,RF-7}
{Los usuarios deben ser capaces de iniciar sesión}
{	\begin{itemize}
	\def\labelenumi{\arabic{enumi}.}
	\tightlist
	\item El usuario debe tener la sesión iniciada.
	\end{itemize}}
{
	\begin{itemize}
	\def\labelenumi{\arabic{enumi}.}
	\tightlist

    \item El usuario hace clic en el botón de finalizar sesión.
    \end{itemize}
}
{El usuario será redirijo a su página de perfil correspondiente}
{}
{Alta}

\cu{3}{Visualizar evolución}
{RF-14}
{El paciente y el médico han de ser capaces de visualizar la evolución del paciente}
{	\begin{itemize}
	\def\labelenumi{\arabic{enumi}.}
	\tightlist
	\item El paciente debe estar dado de alta en la base de datos.
	\item El paciente debe de tener al menos 2 vídeos para ver la gráfica.
	\end{itemize}}
{
	\begin{itemize}
	\def\labelenumi{\arabic{enumi}.}
	\tightlist

    \item El usuario accede a su perfil.
    \item Se visualiza la gráfica de la evolución del paciente.
    \end{itemize}
}
{Se muestra al usuario su gráfico correspondiente}
{}
{Alta}

\cu{4.1}{Dar de alta a usuarios}
{RF-8}
{El administrador debe poder dar de alta a nuevos usuarios.}
{	\begin{itemize}
	\def\labelenumi{\arabic{enumi}.}
	\tightlist
	\item El usuario no debe estar dado de alta.
	\end{itemize}}
{
	\begin{itemize}
	\def\labelenumi{\arabic{enumi}.}
	\tightlist

    \item El administrador pulsa el botón de <<añadir usuario>>.
    \item El administrador introduce los datos del paciente.
    \item El administrador confirma los datos y pulsa el botón <<añadir>>.
    \end{itemize}
}
{Se muestra el nuevo usuario en la lista de usuarios}
{\begin{itemize}
	\def\labelenumi{\arabic{enumi}.}
	\tightlist

    \item El usuario ya existe.
    \item Los datos introducidos son erróneos.
    \end{itemize}}
{Alta}

\cu{4.2}{Dar de baja a usuarios}
{RF-9}
{El administrador debe poder dar de baja a usuarios.}
{	\begin{itemize}
	\def\labelenumi{\arabic{enumi}.}
	\tightlist
	\item El usuario debe estar dado de alta.
	\end{itemize}}
{
	\begin{itemize}
	\def\labelenumi{\arabic{enumi}.}
	\tightlist

    \item El administrador pulsa el botón de <<eliminar usuario>>.
    \item El administrador confirma el usuario y pulsa el botón <<eliminar>>.
    \end{itemize}
}
{Ya no se muestra al usuario en la lista de usuarios}
{\begin{itemize}
	\def\labelenumi{\arabic{enumi}.}
	\tightlist

    \item El usuario no existe.
    \end{itemize}}
{Alta}

\cu{4.3}{Modificar usuarios}
{RF-10}
{El administrador debe poder modificar usuarios.}
{	\begin{itemize}
	\def\labelenumi{\arabic{enumi}.}
	\tightlist
	\item El usuario debe estar dado de alta.
	\end{itemize}}
{
	\begin{itemize}
	\def\labelenumi{\arabic{enumi}.}
	\tightlist

    \item El administrador pulsa el botón de <<modificar usuario>>.
    \item El administrador realiza los cambios del usuario y pulsa el botón <<confirmar>>.
    \end{itemize}
}
{Se muestra la modificación del usuario.}
{\begin{itemize}
	\def\labelenumi{\arabic{enumi}.}
	\tightlist

    \item El usuario modificado coincide con otro que ya existe.
    \end{itemize}}
{Alta}

\cu{4.4}{Visualizar usuarios}
{RF-11}
{El administrador debe poder visualizar la vista de usuarios.}
{	\begin{itemize}
	\def\labelenumi{\arabic{enumi}.}
	\tightlist
	\item Debe existir al menos 1 usuario.
	\end{itemize}}
{
	\begin{itemize}
	\def\labelenumi{\arabic{enumi}.}
	\tightlist

    \item El administrador visualiza la lista de usuarios.
    \end{itemize}
}
{Se muestra la lista de usuarios.}
{}
{Media}

\cu{5.1}{Añadir vídeos}
{RF-2}
{Tanto los médicos como los pacientes han de ser capaces de añadir nuevos vídeos.}
{	\begin{itemize}
	\def\labelenumi{\arabic{enumi}.}
	\tightlist
	\item Debe existir el paciente al que se le quiere añadir el vídeo.
	\end{itemize}}
{
	\begin{itemize}
	\def\labelenumi{\arabic{enumi}.}
	\tightlist

    \item El usuario pulsa el botón <<añadir vídeo>>.
    \item Se rellenan los datos correspondientes del vídeo como la mano.
    \item Se añade el fichero con el vídeo que se quiere subir.
    \item Se pulsa el botón <<añadir>>.
    \end{itemize}
}
{El nuevo vídeo debe estar en la lista de vídeos del paciente correspondiente.}
{\begin{itemize}
	\def\labelenumi{\arabic{enumi}.}
	\tightlist
	\item No se han rellenado los datos correctamente.
	\item El fichero subido no corresponde con formato solicitado.
	\end{itemize}}
{Alta}

\cu{5.2}{Eliminar vídeos}
{RF-3}
{Tanto los médicos como los pacientes han de ser capaces de eliminar vídeos.}
{	\begin{itemize}
	\def\labelenumi{\arabic{enumi}.}
	\tightlist
	\item Debe existir el paciente al que se le quiere eliminar el vídeo.
	\item El paciente debe tener al menos un vídeo.
	\end{itemize}}
{
	\begin{itemize}
	\def\labelenumi{\arabic{enumi}.}
	\tightlist

    \item El usuario pulsa el botón <<eliminar vídeo>> al lado del vídeo que desea eliminar.
    \item Se pulsa el botón <<confirmar>>.
    \end{itemize}
}
{El vídeo ya no esta en la lista de vídeos del paciente correspondiente.}
{}
{Media}

\cu{5.3}{Modificar vídeos}
{RF-4}
{Tanto los médicos como los pacientes han de ser capaces de modificar datos de los vídeos del paciente.}
{	\begin{itemize}
	\def\labelenumi{\arabic{enumi}.}
	\tightlist
	\item Debe existir el paciente al que se le quiere modificar el vídeo.
	\item El paciente debe tener al menos un vídeo.
	\end{itemize}}
{
	\begin{itemize}
	\def\labelenumi{\arabic{enumi}.}
	\tightlist

    \item El usuario pulsa el botón <<editar vídeo>> al lado del vídeo que desea modificar.
    \item El usuario modifica los datos que sean necesarios del vídeo seleccionado.
    \item El usuario pulsa el botón <<confirmar>>.
    
    \end{itemize}
}
{El vídeo debe aparecer con las modificaciones realizadas en la lista de vídeos del paciente correspondiente.}
{}
{Baja}

\cu{5.4}{Clasificar vídeos}
{RF-5}
{Tanto los médicos como los pacientes han de ser capaces de clasificar los vídeos del paciente.}
{	\begin{itemize}
	\def\labelenumi{\arabic{enumi}.}
	\tightlist
	\item Debe existir el paciente al que se le quiere clasificar el vídeo.
	\item El paciente debe tener al menos un vídeo.
	\end{itemize}}
{
	\begin{itemize}
	\def\labelenumi{\arabic{enumi}.}
	\tightlist

    \item El usuario pulsa el botón <<clasificar vídeo>>.
    \item Se clasificará el vídeo correspondiente.

    
    \end{itemize}
}
{Aparecerá la clasificación del vídeo.}
{}
{Alta}

\cu{6.1}{Añadir paciente}
{RF-12}
{El médico a de ser capaz de añadir pacientes a su lista de pacientes.}
{	\begin{itemize}
	\def\labelenumi{\arabic{enumi}.}
	\tightlist
	\item Debe existir el paciente que se desea añadir.
	\end{itemize}}
{
	\begin{itemize}
	\def\labelenumi{\arabic{enumi}.}
	\tightlist

    \item El médico pulsa el botón <<añadir paciente>>.
    \item Se deberá rellenar los datos del paciente que se desea agregar.
    \item El paciente se agregará a la lista de pacientes.
    
    \end{itemize}
}
{Aparecerá el paciente en la lista del médico.}
{	\begin{itemize}
	\def\labelenumi{\arabic{enumi}.}
	\tightlist

    \item El paciente no existe.
    \item El paciente ya se encuentra en la lista del médico.
    
    \end{itemize}}
{Alta}

\cu{6.2}{Eliminar paciente}
{RF-13}
{El médico a de ser capaz de eliminar pacientes a su lista de pacientes.}
{	\begin{itemize}
	\def\labelenumi{\arabic{enumi}.}
	\tightlist
	\item Debe existir el paciente que se desea eliminar.
	\item El paciente debe estar incluido en la lista de pacientes del médico.
	\end{itemize}}
{
	\begin{itemize}
	\def\labelenumi{\arabic{enumi}.}
	\tightlist

    \item El médico pulsa el botón <<eliminar paciente>>.
    \item Se selecciona el paciente que se desea eliminar.
	\item Se confirma que se desea eliminar a ese paciente.
    
    \end{itemize}
}
{Ya no aparecerá el paciente en la lista del médico.}
{	\begin{itemize}
	\def\labelenumi{\arabic{enumi}.}
	\tightlist

    \item El paciente no existe.
    \item El paciente no se encuentra en la lista del médico.
    
    \end{itemize}}
{Media}


