\apendice{Anexo de sostenibilización curricular}

\section{Introducción}
Este apartado, aborda los diversos aspectos de sostenibilidad aplicados durante el desarrollo del proyecto.
Para la elaboración de este apartado, se ha utilizado como referencia el documento <<Directrices para la inclusión de la sostenibilidad en el currículo universitario>> publicado por la Conferencia de Rectores de las Universidades Españolas (CRUE)\footnote{\url{https://www.crue.org/wp-content/uploads/2020/02/Directrices_Sosteniblidad_Crue2012.pdf}}. Este documento proporciona un marco teórico y práctico para integrar la sostenibilidad en los proyectos educativos y de investigación. 

Este apartado se divide en tres secciones: impacto del software a nivel de consumo, igualdad y accesibilidad y conclusiones.

\section{Impacto del software a nivel de consumo}
El impacto del software en términos de consumo energético y de recursos es un aspecto crítico a considerar en el desarrollo de aplicaciones sostenibles. Durante el proyecto se ha hecho uso de modelos de aprendizaje y servidores.

\subsection{Servidor}

Para el despliegue de la aplicación se ha hecho uso del servidor de la UBU. Este servidor genera un consumo constante debido a la necesidad de mantener la infraestructura para poder proporcionar servicios de manera ininterrumpida. Es este uso continuo el que lleva a un consumo energético elevado. Para solucionar esto se podría hacer un estudio para mejorar los procesadores y componentes del servidor para ser más eficientes energéticamente o apagar aquellos componentes y servicios que no estén en uso.

\subsection{Aprendizaje automático}

Los modelos de aprendizaje automático consumen una cantidad significativa de energía por varias razones, relacionadas principalmente con el proceso de entrenamiento y la infraestructura necesaria para ejecutar estos modelos. Debido a que el entrenamiento de modelos, especialmente los de deep learning, requiere realizar cálculos complejos y numerosas iteraciones sobre grandes conjuntos de datos y las unidades de procesamiento gráfico (GPU) y las unidades de procesamiento tensorial (TPU) son esenciales para el entrenamiento eficiente, pero también son muy intensivas en energía.

Como solución se propone el uso de modelos más pequeños y eficiente, entrenamiento distribuido o adquisición de \textit{hardware} específico. Otras soluciones podría ser entrenamiento fuera de las horas pico programando el entrenamiento de modelos durante períodos de baja demanda energética u optimización de la carga de trabajo balanceando y distribuyendo la carga de trabajo de manera óptima para evitar picos de consumo energético.
El desarrollo de este software también ha considerado la sostenibilidad a largo plazo, buscando minimizar su huella de carbono. Se han implementado prácticas de programación eficientes, optimizando el código para reducir el tiempo de procesamiento y, por ende, el consumo energético.

\section{Igualdad y accesibilidad}
Uno de los objetivos fundamentales del software es su accesibilidad para todos los usuarios, independientemente de sus capacidades o limitaciones. Este enfoque inclusivo garantiza que el software sea una herramienta útil y eficaz para el mayor número posible de personas, alineándose con los principios de igualdad y no discriminación.

El compromiso con la igualdad también se refleja en la estructura de precios del software. Dado que el proyecto está destinado a facilitar el trabajo de pacientes y médicos, se ha decidido que su uso sea completamente gratuito. Esta decisión garantiza que el software esté disponible para todos los usuarios, independientemente de su situación económica, promoviendo así la igualdad de oportunidades .

\section{Conclusiones}
La integración de la sostenibilidad en el desarrollo de software es esencial para minimizar el impacto ambiental y promover la igualdad de acceso a la tecnología. Este proyecto ha sido diseñado con un enfoque holístico, considerando tanto la eficiencia energética como la accesibilidad y la igualdad. Siguiendo las directrices de sostenibilidad de la CRUE, se ha logrado desarrollar un software que no solo cumple con sus objetivos funcionales, sino que también contribuye a un futuro más sostenible y equitativo.





