\apendice{Especificación de diseño}

\section{Introducción}
En el siguiente apartado se describen los detalles de la estructura del \textit{software}: el comportamiento y la interacción entre los distintos componentes.

\section{Diseño de datos}
Este diseño se ha dividido en varias fases: diseño del diagrama entidad-relación, diseño del diagrama relacional y diseño del diccionario de datos.

\subsection{Modelo entidad-relación}

Para representar gráficamente las relaciones entre los datos, se ha creado un diagrama E-R (ver figura~\ref{fig:./img/DiagramaER.pdf}). En el se puede observar una ISA exclusiva total. Esto se debe a que, tras consultarlo con el profesor Jesús Maudes, se llegó a la conclusión de que crear una ISA exclusiva total era una solución válida debido a que se quiere evitar que un mismo usuario tenga acceso a varias funcionalidades. En caso de que esto se requiriese (si un doctor necesita permisos de administración) se crearía otra cuenta de forma más controlada.
\begin{landscape}
\imagenAmpliada{./img/DiagramaER.pdf}{Diagrama entidad-relación}
\end{landscape}
\subsection{Modelo relacional}

Partiendo del modelo entidad-relación (ver figura~\ref{fig:./img/DiagramaER.pdf}) se ha creado el diagrama relacional (ver figura~\ref{fig:./img/DiagramaRelacional.pdf}).
A continuación se mencionan algunas de las decisiones más importantes que se han decido tomar a la hora de crear este diagrama.
Con respecto a la ISA se ha decidido dividir en tres tablas distintas ya que en un principio un usuario solo va a poder tener uno de los tres roles asignados y en caso de que se necesitase que un doctor tuviera el rol de administrador, que es el un caso excepcional, se le otorgaría otra cuenta distinta con los permisos de administración. De esta manera podremos controlar los accesos de cada usuario de manera más limpia y exclusiva.
El rol de administrador es un rol que pese a que no se relacione con ninguna otra entidad, se ha visto que es necesario para poder permitir el acceso a los controles de administrador.
\imagen{./img/DiagramaRelacional.pdf}{Diagrama relacional de la aplicación}

\subsection{Diccionario de datos}
Se adjunta a continuación, para cada tabla, el diccionario de datos correspondiente.
\begin{itemize}
\item \textbf{Administradores}: la tabla~\ref{datadic:admin} contiene la información correspondiente a los administradores de la aplicación.
\begin{table}
	\scalebox{0.80}{
		\begin{tabular}{@{}p{5em} p{6em} p{6em} p{20em}@{}}
			\toprule
			\textbf{Nombre} & \textbf{Tipo} & \textbf{Columna} & \textbf{Descripción}\\
			\midrule
			\texttt{id} & \texttt{INTEGER} & \texttt{\textbf{PK}} & Identificador único del administrador. \\
			\texttt{username} & \texttt{VARCHAR(100)} & \texttt{UNIQUE NOT NULL} & Nombre de usuario del administrador. Debe ser único. \\
			\texttt{password} & \texttt{VARCHAR(100)} & \texttt{NOT NULL} & Contraseña del administrador, cifrada usando sha-256. \\
			\texttt{first\_name} & \texttt{VARCHAR(100)} & \texttt{NOT NULL} & Primer nombre del administrador. \\
			\texttt{last\_name} & \texttt{VARCHAR(100)} & \texttt{NOT NULL} & Apellidos del administrador. \\
			\bottomrule
		\end{tabular}
	}
	\caption[Diccionario de datos: \texttt{Admin}]{Diccionario de datos. Tabla correspondiente a la clase \texttt{Admin}.}
	\label{datadic:admin}
\end{table}
\item \textbf{Doctores}: la tabla~\ref{datadic:doctor} contiene la información correspondiente a los doctores.
\begin{table}
	\scalebox{0.80}{
		\begin{tabular}{@{}p{5em} p{6em} p{6em} p{20em}@{}}
			\toprule
			\textbf{Nombre} & \textbf{Tipo} & \textbf{Columna} & \textbf{Descripción}\\
			\midrule
			\texttt{id} & \texttt{INTEGER} & \texttt{\textbf{PK}} & Identificador único del doctor. \\
			\texttt{username} & \texttt{VARCHAR(100)} & \texttt{UNIQUE NOT NULL} & Nombre de usuario del doctor. Debe ser único. \\
			\texttt{password} & \texttt{VARCHAR(100)} & \texttt{NOT NULL} & Contraseña del doctor, cifrada usando sha-256. \\
			\texttt{first\_name} & \texttt{VARCHAR(100)} & \texttt{NOT NULL} & Primer nombre del doctor. \\
			\texttt{last\_name} & \texttt{VARCHAR(100)} & \texttt{NOT NULL} & Apellidos del doctor. \\
			\bottomrule
		\end{tabular}
	}
	\caption[Diccionario de datos: \texttt{Doctor}]{Diccionario de datos. Tabla correspondiente a la clase \texttt{Doctor}.}
	\label{datadic:doctor}
\end{table}
\item \textbf{Pacientes}: la tabla~\ref{datadic:patient} contiene la información correspondiente a los pacientes.
\begin{table}
\scalebox{0.80}{
	\begin{tabular}{@{}p{5em} p{6em} p{6em} p{20em}@{}}
		\toprule
		\textbf{Nombre} & \textbf{Tipo} & \textbf{Columna} & \textbf{Descripción}\\
		\midrule
		\texttt{id} & \texttt{INTEGER} & \texttt{\textbf{PK}} & Identificador único del paciente. \\
		\texttt{username} & \texttt{VARCHAR(100)} & \texttt{UNIQUE NOT NULL} & Nombre de usuario del paciente. Debe ser único. \\
		\texttt{password} & \texttt{VARCHAR(100)} & \texttt{NOT NULL} & Contraseña del paciente, cifrada usando sha-256. \\
		\texttt{first\_name} & \texttt{VARCHAR(100)} & \texttt{NOT NULL} & Primer nombre del paciente. \\
		\texttt{last\_name} & \texttt{VARCHAR(100)} & \texttt{NOT NULL} & Apellidos del paciente. \\
		\texttt{birth\_date} & \texttt{DATE} & \texttt{NOT NULL} & Fecha de nacimiento del paciente. \\
		\texttt{handedness} & \texttt{ENUM('right', 'left')} & \texttt{NOT NULL} & Mano dominante del paciente (derecha o izquierda). \\
		\texttt{gender} & \texttt{ENUM('M', 'F')} & \texttt{NOT NULL} & Género del paciente (masculino o femenino). \\
		\texttt{doctor\_id} & \texttt{INTEGER} & \texttt{FK(doctor.id)} & Clave foránea a la tabla Doctor. Indica el doctor a cargo del paciente. \\
		\bottomrule
	\end{tabular}
	}
\caption[Diccionario de datos: \texttt{Patient}]{Diccionario de datos. Tabla correspondiente a la clase \texttt{Patient}.}
\label{datadic:patient}
\end{table}
\item \textbf{Vídeos}: la tabla~\ref{datadic:vídeo} contiene la información correspondiente a los vídeos de los pacientes pacientes.
\begin{table}
	\scalebox{0.80}{
		\begin{tabular}{@{}p{5em} p{6em} p{6em} p{20em}@{}}
			\toprule
			\textbf{Nombre} & \textbf{Tipo} & \textbf{Columna} & \textbf{Descripción}\\
			\midrule
			\texttt{id} & \texttt{INTEGER} & \texttt{\textbf{PK}} & Identificador único del vídeo.\\
			\texttt{patient\_id} & \texttt{INTEGER} & \texttt{FK
			(patient.id)} & Clave foránea a la tabla Paciente. Indica el paciente asociado al vídeo.\\
			\texttt{hand} & \texttt{ENUM('left', 'right')} & \texttt{NOT NULL} & Mano usada en el vídeo (izquierda o derecha). \\
			\texttt{date} & \texttt{DATE} & \texttt{NOT NULL} & Fecha de grabación del vídeo. \\
			\texttt{vídeo\_data} & \texttt{LargeBinary} & \texttt{NOT NULL} & Datos binarios del vídeo. \\
			\texttt{amplitude} & \texttt{VARCHAR(100)} & \texttt{NOT NULL} & Amplitud clasificada para el vídeo. \\
			\texttt{slowness} & \texttt{VARCHAR(100)} & \texttt{NOT NULL} & Lentitud clasificada para el vídeo. \\
			\bottomrule
		\end{tabular}
	}
	\caption[Diccionario de datos: \texttt{vídeo}]{Diccionario de datos. Tabla correspondiente a la clase \texttt{vídeo}.}
	\label{datadic:vídeo}
\end{table}

\item \textbf{Medicinas}: la tabla~\ref{datadic:medicine} contiene la información de las medicinas.
\begin{table}
	\scalebox{0.80}{
		\begin{tabular}{@{}p{5em} p{6em} p{6em} p{20em}@{}}
			\toprule
			\textbf{Nombre} & \textbf{Tipo} & \textbf{Columna} & \textbf{Descripción}\\
			\midrule
			\texttt{id} & \texttt{INTEGER} & \texttt{\textbf{PK}} & Identificador único del medicamento. \\
			\texttt{name} & \texttt{VARCHAR(100)} & \texttt{NOT NULL} & Nombre del medicamento. \\
			\bottomrule
		\end{tabular}
	}
	\caption[Diccionario de datos: \texttt{Medicine}]{Diccionario de datos. Tabla correspondiente a la clase \texttt{Medicine}.}
	\label{datadic:medicine}
\end{table}
\item \textbf{PacienteMedicinas}: la tabla~\ref{datadic:patientmedicine} contiene la información de las medicinas administradas a cada paciente incluyendo la fecha de inicio del tratamiento así como la fecha de finalización.
\begin{table}
	\scalebox{0.80}{
		\begin{tabular}{@{}p{5em} p{6em} p{6em} p{20em}@{}}
			\toprule
			\textbf{Nombre} & \textbf{Tipo} & \textbf{Columna} & \textbf{Descripción}\\
			\midrule
			\texttt{id} & \texttt{INTEGER} & \texttt{\textbf{PK}} & Identificador único de la asociación entre paciente y medicamento. \\
			\texttt{patient\_id} & \texttt{INTEGER} & \texttt{FK
			(patient.id)} & Clave foránea a la tabla Paciente. Indica el paciente que toma el medicamento. \\
			\texttt{medicine\_id} & \texttt{INTEGER} & \texttt{FK
			(medicine.id)} & Clave foránea a la tabla Medicamento. Indica el medicamento que toma el paciente. \\
			\texttt{dosage} & \texttt{VARCHAR(100)} & \texttt{NOT NULL} & Dosis del medicamento para el paciente. \\
			\texttt{start\_date} & \texttt{DATE} & \texttt{NOT NULL} & Fecha de inicio del tratamiento con el medicamento. \\
			\texttt{end\_date} & \texttt{DATE} & \texttt{NULL} & Fecha de fin del tratamiento con el medicamento (opcional). \\
			\bottomrule
		\end{tabular}
	}
	\caption[Diccionario de datos: \texttt{PatientMedicine}]{Diccionario de datos. Tabla correspondiente a la clase \texttt{PatientMedicine}.}
	\label{datadic:patientmedicine}
\end{table}
\end{itemize}
\section{Diseño procedimental}
El diseño procedimental en informática es una metodología de desarrollo de software que se centra en la creación de procedimientos o funciones para llevar a cabo tareas específicas dentro de un programa.
Durante el desarrollo del proyecto se ha hecho uso de la herramienta Draw.io\footnote{Fuente:~\url{draw.io}} para representar los procedimientos. Dentro del sistema podemos distinguir los siguientes componentes:
\begin{itemize}
\item Doctor: representado por el monigote, representa a un usuario que haya iniciado sesión como doctor.
\item Navegador: se refiere al apartado de \textit{frontend} de la aplicación.
\item Servidor web: se refiere a la parte de \textit{backend}.
\item Paddel: se refiere a la librería de Paddel utilizada durante el proyecto.
\item Base de datos: se refiere a la base de datos de la aplicación.
\end{itemize}

Debido a que los diagramas secuenciales de la aplicación son muy básicos exceptuando el de la clasificación de los vídeos, se ha tomado la decisión de omitirlos.

\subsection{Clasificación de los vídeos}
Para esta funcionalidad, el doctor ha de tener al menos un paciente para poder realizar la acción. El doctor es redirigido a la pestaña de añadir vídeos al usuario. Tras recibirla, el doctor envía la información del vídeo junto con el archivo del mismo a la aplicación. La aplicación lo envía a la librería Paddel para extraer las características y estas características se envían a los modelos entrenados para obtener las clasificaciones. Una vez obtenidas el vídeo es guardado en la base de datos y posteriormente, se envía al usuario de vuelta a la página de gestión de vídeos actualizada del paciente al que se le ha añadido.
La figura~\ref{fig:./img/DiagramaSecuencia} muestra el diagrama de secuencia del proceso.
\imagen{./img/DiagramaSecuencia}{Diagrama de secuencia de clasificación de los vídeos}{}


\section{Diseño arquitectónico}


Para ilustrar la arquitectura de la aplicación se emplea el diagrama de despliegue de la figura~\ref{fig:./img/DiagramaDespliegue.png}. Este tipo de diagrama UML muestra cómo están organizados físicamente los componentes del software en el sistema.

La aplicación ha sido desplegada en los servidores de la UBU.

El Data Base Server se encarga de gestionar la capa de persistencia del proyecto. Para ello, se ha utilizado SQLite como herramienta de gestión de la base de datos.

El usuario o cliente actúa como el controlador de la interfaz de usuario, interactuando con las páginas HTML a través de su navegador.
\imagen{./img/DiagramaDespliegue.png}{Diagrama de despliegue}{}
