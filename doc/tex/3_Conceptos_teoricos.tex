\capitulo{3}{Conceptos teóricos}

En este capítulo se definirán algunos conceptos teóricos para facilitar la compresión de este proyecto.

\section{Enfermedad del Parkinson}

En este apartado se explicarán lo conceptos teóricos relativos a la enfermedad del Parkinson (EP).
\subsection{Concepto general de la EP}
La EP es una  enfermedad neurodegenerativa multisistémica progresiva que cada afecta principalmente a a gente de avanzada edad~\cite{pdsymptoms}.  
Entre los síntomas principales de esta enfermedad podemos encontrar:
Perdida significativa de parte de las células productoras de dopamina lo que produce que aparezcan mucho antes los síntomas relacionados con el movimiento que presenta la EP.Otro de los síntomas que podemos encontrar en la EP es la bradicinesia es cual se caracteriza por la lentitud al realizar movimientos voluntarios así como la ralentización y decremento de amplitud a la hora de realizar movimientos repetitivos.
El diagnóstico de la EP se basa en criterios específicos. Los síntomas iniciales incluyen lentitud de movimientos, junto con rigidez muscular, temblores o problemas de equilibrio. Los controles posteriores descartan otras posibles causas y confirman la presencia de factores que sugieren claramente la EP.
Progresión habitual: La EP suele empezar en un lado del cuerpo y extenderse a lo largo de unos años. Los síntomas incluyen postura encorvada, rigidez, letra más pequeña y marcha arrastrando los pies. Los temblores son frecuentes.
Otros problemas: Las alteraciones de la marcha (como pasos indecisos o congelación repentina) se vuelven habituales. La pérdida de equilibrio es un problema importante, que aumenta el riesgo de caídas y lesiones.

\subsection{Rapid finger tapping test}

El test de golpeteo rápido de los dedos, también conocido como <<rapid finger tapping test>> o RFTT es un procedimiento por el cual el paciente realiza golpeteos de manera repetitiva durante un periodo de entre 10 y 15 segundos en los cuales ha de intentar generar la mayor amplitud posible entre el dedo indice y el pulgar sin bajar la frecuencia a la que lo realiza. Este test es usado comunmente para el diagnostico de personas con la EP ya que como se explicó anteriormente uno de los síntomas que presenta esta enfermedad el la bradicisnesia el cual produce que una persona con EP al realizar este test muestre deteriodo ya sea en la amplitud o en la frecuencia según avanza la prueba. Este test será el que se utilizará en este proyecto para determinar el grado de EP en el que se encuentra el paciente.

\subsection{Unified parkinson disease rating scale}
